\documentclass{DIKU-report}

\setlength{\errorcontextlines}{999} %Can be used for debugging purposes

\usepackage{amsmath}
\usepackage{color}
\usepackage{graphicx}   
\usepackage{hyperref}
\usepackage{longtable}
\usepackage[utf8]{inputenc}
\usepackage{url}

\newcommand{\Code}[1]{\mbox{\color{blue}\texttt{#1}}}
\newcommand{\twodots}{\,.\,.\,}

\usepackage{listings,lstmisc}

\lstdefinestyle{customC++}{
  breaklines=true,
  basewidth={0.5em,0.45em},
  basicstyle=\small,
  identifierstyle=\ttfamily,
  commentstyle=\itshape,
  aboveskip=0.5 \baselineskip,
  belowskip=0.0 \baselineskip,
  xleftmargin=12pt,
  language=C++,
  numbers=left, 
  numberstyle=\tiny, 
  stepnumber=1, 
  numbersep=5pt,
  tabsize=2,
  morekeywords={constexpr,define,defined,endif,elif,ifdef,ifndef,include,noexcept,nullptr},
  keywordstyle=\bfseries,
}
\lstset{mathescape} % Allow escaping to LaTeX inside $..$
\lstset{literate=*
  {!}{{$\mbox{\textbf{not}}$}}3
}

\lstset{literate=
  {-}{{\texttt{-}}}1 
  {+}{{\texttt{+}}}1 
  {--}{{\texttt{-}\hspace*{0.05em}\texttt{-}}}2 
  {++}{{\texttt{+}\texttt{+}}}2
  {=}{{\texttt{=}}}1
  {+=}{{\texttt{+}\texttt{=}}}2 
  {-=}{{\texttt{-}\hspace*{0.05em}\texttt{=}}}2 
%  {&}{{\texttt{\&}}}1
  {[]}{{\texttt{[}$\,$\texttt{]}}}2
  {\#}{\# }2
  {\%}{\% }2
  {infty}{$\infty$}1
  {©}{\copyright{}}1 
  {§}{\textit{\S}$\!$}1 
  {ø}{{{\o}}$\!$}1 
  {ä}{{{\"a}}$\!$}1
  {^2}{${}^{2}$}1 
  {.}{{$\mathrm{\textbf{.}}\hspace*{0.05em}$}}{1}
  {->}{{$\rightarrow$}}2
  {"}{\texttt{"}}1 
  {~}{$\sim$}{2}
  {|}{{$\mid$}}1     
%  {&}{$\mathtt{\&}\!$}2
  {&}{{$\boldsymbol{\&}$}}2
  {&&}{{\hspace*{0.1em}$\boldsymbol{\&}\boldsymbol{\&}$}}3
  {()}{{$(\,)$}}2
  {==}{{$=\!=$}}2
  {!=}{{$=\!\!\!\neq\!\!\!=$}}2
  { || }{$\,\,\mbox{\textbf{or}}\,\,$}2
  {>=}{$\geq$}1
  {<=}{$\leq$}1
  {>>=}{$>$$>$\,\,$=$}3
  {---}{{\rm ---}}2
  {ell}{{$\ell$}}1
}
\lstset{style=customC++}

\titlehead{Multiplication Benchmark}
\authorhead{\{Marius-Florin Cristian; Sidharth Singhal\}}

\title{Multiplication Horse-Race}

\author{Marius-Florin Cristian; Sidharth Singhal}
 
\institute{%
Department of Computer Science, University of Copenhagen\\
Universitetsparken 5, 2100 Copenhagen East, Denmark\\ 
\texttt{wdx186@alumni.ku.dk; thx889@alumni.ku.dk}}

\begin{document}

\begin{titlepage}

\maketitle

\begin{abstract}
This progress report contains the implementation and benchmarks for different multiplication algorithms (naive, karatsuba, strassen, furer).   
\end{abstract}

\begin{keywords}
Multiplication, naive, Karatsuba, Sch\"onhage-Strassen, F\"urer, large integers, arbitrarly large digits, Fast-Fourier Transform
\end{keywords}

\dates{ 2017, October 2017.}

\end{titlepage}
\vspace*{\fill}
\newpage
\addvspace{\bigskipamount}
\section*{Index}

\vspace*{\fill}
\newpage
\addvspace{\bigskipamount}
\section*{Introduction}
\subsection*{Motivation}

\subsection*{Goals}
The Goals of this project are to benchmark 
\vspace*{\fill}
\newpage
\addvspace{\bigskipamount}
\section*{Approach}
In our initial approach, we tried to work with the metaprogramming template provided by Jyrki, but we failed to cast data types from $Z<n>$ to $Z<m>$;\\
Therfore we had a change of dirrection and switched to an approach more python like: the inputs are treated as strings thus we can read around bilion digit numbers;\\
The first problem we encountered was efficiently storing the result in memory; we initially chose a $std::vector<int>$ to build our result, where one cell represents one digit. The major drawback is obvious as one digit can have values from 0 to 9; thus occupying at most 4 bits. Thus our results were capped at around 8000 digits output before running out of memory.\\
To overcome this problem, we chose to encode our one base 10 digit on a nibble; and to store 2 digits on an unsigned char. The problem is that $char$ data type does not support multiplication. Thus we used $uint16\_t$ data structure to store 4 digits (4 nibbles). The drawback of this approach is, that on 4 bits we can represent values up to 15, and we only represent up to 9.\\
\url{https://en.wikipedia.org/wiki/Double_dabble}\\
\url{https://en.wikipedia.org/wiki/Multiplication_algorithm#Shift_and_add}\\
\url{https://en.wikipedia.org/wiki/Booth%27s_multiplication_algorithm}\\
By using this encoding our result can be around 260000 digits (32.5 times larger input than the initial phase); but everything that exceeds will just be ignored; Not entirerly satisfactory as we expected to do milion digit multiplication and see the turning points in performance. Further tuning can be made; as on 8 bits we store values up to 99 instead of 256.\\
A second tought: instead of having cell size of $uint16\_t$ we can use $long\ long$, will this result in an improvment on the amount of difits we can use?\\
A third tought (as suggested by a peer): why not have a lookup table for cell multiplication values, and do it in O(1); this table will take up memory space, it might make it faster for instances up to a number of digits, but the trade off will be in the length of the result.\\
A forth tought: $x^{z}$ is computed much faster than $x_1x_2...x_z$ (exponentiation by squaring); having a lookup table to break the cell in power of primes, will this yeld in better performance? \\
\url{https://www.rookieslab.com/posts/fast-power-algorithm-exponentiation-by-squaring-cpp-python-implementation}

\vspace*{\fill}
\newpage
\addvspace{\bigskipamount}
\section*{Experiments}
\subsection*{System Info}
\lstinputlisting[caption=Machine 1,label=dmi_proc_m1,language=,]{listings/dmidecode_processor}
\lstinputlisting[caption=Machine 2,label=dmi_proc_m1,language=,]{listings/dmidecode_processor}

\subsection*{Instances}
\subsection*{Design}
\subsection*{Measures}
\subsection*{Factors}
\vspace*{\fill}
\newpage
\addvspace{\bigskipamount}
\section*{Conclusions}

%
%\begin{bibliography}{1}

%\bibitem{recent} FAST INTEGER MULTIPLICATION USING GENERALIZED FERMAT PRIMES - SVYATOSLAV COVANOV AND EMMANUEL THOMÉ \url{https://arxiv.org/pdf/1502.02800.pdf}

%\end{bibliography}
\end{document}
